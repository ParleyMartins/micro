\documentclass[conference]{IEEEtran}
\usepackage[utf8x]{inputenc}
\usepackage[portuguese]{babel}
\hyphenation{op-tical net-works semi-conduc-tor}


\begin{document}
\title{Automação residencial: Comandando lâmpadas pelo \emph{Telegram}}


\author{\IEEEauthorblockN{Parley Martins}
\IEEEauthorblockA{Faculdade UnB Gama\\
Universidade de Brasília\\
Brasília, Brasil\\
parleypachecomartins@gmail.com}
}

\maketitle


\begin{abstract}
%\boldmath
Este trabalho propõe a utilização de automação residencial para possibilitar ao usuário acender e apagar lâmpadas remotamente, utilizando integração com aplicativo no celular.
\end{abstract}

\IEEEpeerreviewmaketitle

\section{Introdução}
% justificativa, objetivos, requisitos, benefícios, revisão bibliográfica).
Automação residencial é resultado da combinação de espaços residenciais, como sala, banheiro, quarto com tecnologias, para maior conforto, segurança, ou menos contato humano \cite{moraes2001using}. Estas tecnologias e ideias eram, até recentemente, consideradas sonhos de um futuro distante \cite{GHAFFARIANHOSEINI2013593}, sem uso prático, exceto no entretenimento.

No entanto com um mundo conectado pela internet, que mudou o jeito que as pessoas se comunicam e se relacionam, é normal que este conceito esteja cada vez mais próximo da realidade das pessoas. Para ter mais conforto, já é possível controlar pelo celular o volume das televisões (e outros aparelhos de som), o canal em que se está, a intensidade com que aparelhos devem funcionar, entre outras comodidades. Para ter mais segurança, é possível controlar luzes, sistema de alarmes, de detecção de movimentos, etc. Existem diversas empresas que fornecem esse tipo de serviço, mas eles ainda podem ter um custo muito elevado.

\section{Solução}
\subsection{Hardware}
\subsection{Software}
\section{Resultados}
\section{Conclusão}

\bibliographystyle{IEEEtran}
\bibliography{IEEEabrv,bibliografia}

\end{document}


