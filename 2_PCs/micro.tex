\documentclass[12pt,journal,compsoc]{IEEEtran}
\usepackage[utf8x]{inputenc}
\usepackage[portuguese]{babel}
\hyphenation{op-tical net-works semi-conduc-tor}

\begin{document}
\title{Automação residencial:\\Comandando lâmpadas pelo \emph{Telegram}}

\author{\IEEEauthorblockN{Parley Martins - 11/0038096}
% \IEEEauthorblockA{Faculdade UnB Gama\\
% Universidade de Brasília\\
% Brasília, Brasil\\
% parleypachecomartins@gmail.com}
}

\IEEEcompsoctitleabstractindextext{%
\begin{abstract}
%\boldmath
Este trabalho propõe a utilização de automação residencial para possibilitar ao usuário acender e apagar lâmpadas remotamente, utilizando integração com aplicativo no celular.
\end{abstract}

\begin{IEEEkeywords}
Automação Residencial, Telegram, bot, \emph{smart house}
\end{IEEEkeywords}}

\maketitle

\section{Introdução}
% justificativa, objetivos, requisitos, benefícios, revisão bibliográfica).
Automação residencial é resultado da combinação de espaços residenciais, como sala, banheiro, quarto com tecnologias, para maior conforto, segurança, ou menos contato humano \cite{moraes2001using}. Estas tecnologias e ideias eram, até recentemente, consideradas sonhos de um futuro distante \cite{GHAFFARIANHOSEINI2013593}, sem uso prático, exceto no entretenimento.

No entanto com um mundo conectado pela internet, que mudou o jeito que as pessoas se comunicam e se relacionam, é normal que este conceito esteja cada vez mais próximo da realidade das pessoas. Para ter mais conforto, já é possível controlar pelo celular o volume das televisões (e outros aparelhos de som), o canal em que se está, a intensidade com que aparelhos devem funcionar, entre outras comodidades. Para ter mais segurança, é possível controlar luzes, sistema de alarmes, de detecção de movimentos, etc. Existem diversas empresas que fornecem esse tipo de serviço, mas eles ainda podem ter um custo muito elevado.

\section{Solução}

Para facilitar e desmistificar o acesso à automação residencial, a proposta deste projeto é implementar um sistema que possa controlar remotamente as lâmpadas de uma casa. O usuário, após instalação do sistema físico, poderá utilizar seu \emph{smartphone} para ligar e desligar as lâmpadas.

A interação com o usuário se dará através de um bot no aplicativo \emph{Telegram}. Deve-se iniciar uma 'conversa' com o bot, e mandar o comando desejado (ligar ou desligar, por exemplo). Este irá mandar para o módulo wifi do sistema, que fará a comunicação com o MSP, desligando ou ligando a lâmpada selecionada.

Para fins deste trabalho, uma lâmpada e uma fonte de energia externas, controladas pela protoboard, serão utilizadas para facilitar a instalação e testes.

O hardware será composto, inicialmente, pelos seguintes items:

\begin{itemize}
\item protoboard, para execução do sistema;
\item microcontrolador MSP430, irá executar o controle da energia na lâmpada;
\item módulo esp8266, proverá o acesso à rede wifi;
\item lâmpada, para testes;
\item fonte de energia, tanto para o microcontrolador quanto para a lâmpada.
\end{itemize}

\subsection{Requisitos}

O software do microcontrolador deve corretamente identicar os comandos e apagar ou acender a lâmpada, conforme instrução recebida.

O \emph{bot}, software que responde a comandos pré definidos automaticamente, deve ser integrado ao aplicativo \textit{Telegram} e deve mandar instruções de ligar e de desligar a lâmpada.

O sistema, tanto harware quanto software, deve ter acesso à internet para o funcionamento correto.


\bibliographystyle{IEEEtran}
\bibliography{IEEEabrv,bibliografia}

\end{document}


